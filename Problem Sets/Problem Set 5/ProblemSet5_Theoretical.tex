\documentclass{article}

%----------------------------------------------------------------------------------------
%	FONTS AND ENCODINGS
%----------------------------------------------------------------------------------------
\usepackage[utf8]{inputenc} % Required for inputting international characters
\usepackage[T1]{fontenc} % Output font encoding for international characters
\usepackage[english]{babel} % language pack..
\usepackage{amsmath} % math symbols
\usepackage{lastpage}
\usepackage{amssymb} % math symbols
\usepackage{cmbright} % font
\usepackage{pifont}
\usepackage{graphicx,float} % for pictures etc.
\usepackage{subcaption}
\usepackage{tcolorbox}
\usepackage{ulem}
\usepackage{cancel}
\usepackage{array}
\usepackage{xcolor}
\usepackage{slashed}
\usepackage{multirow}
\usepackage{wasysym}
\usepackage{minted}
\usepackage[hidelinks]{hyperref}
\usepackage{subfiles}

%----------------------------------------------------------------------------------------
%	PAGE LAYOUT
%----------------------------------------------------------------------------------------
\setlength{\parindent}{0bp} % No paragraph indentation
\usepackage{fancyhdr} % Headings/footers

\usepackage{geometry} % Required for adjusting page dimensions and margins
\geometry{
	paper=a4paper, % Paper size, change to letterpaper for US letter size
	top=2cm, % Top margin
	bottom=2cm, % Bottom margin
	left=3cm, % Left margin
	right=3cm, % Right margin
	inner=2cm,
	outer=2cm,
	headheight=2cm, % Header height
	footskip=1cm, % Space from the bottom margin to the baseline of the footer
	headsep=0.75cm, % Space from the top margin to the baseline of the header
}

\fancyhead{} % Clear all headers
\fancyfoot{} % Clear all footers
\fancyfoot[R]{\footnotesize \thepage\ / \pageref{LastPage}}
\fancyhead[L]{} %Left
\fancyhead[C]{} %Center
\fancyhead[R]{} %Right

\renewcommand{\headrulewidth}{0pt} % Remove header rule
\renewcommand{\footrulewidth}{0pt} % Remove footer rule
\pagestyle{fancy} % Set page style to "fancy"


%----------------------------------------------------------------------------------------
%	DOCUMENT
%----------------------------------------------------------------------------------------

\begin{document}

\section{Problem Set 5 - Problem 1)}

\subsection{Derive GAS model}

We have,
\[
Y_{t}\rvert\mathcal{F}_{t-1}\sim\mathcal{T}\left(0,\phi,\nu\right)
\]

where the mean is $0$, $\phi$ is the scale parameter and $\nu$
is degrees of freedom. Remembering that the $\mathcal{T}$ distribution
is defined not by std. deviation but by scale. However the following
relation is true
\[
\sigma_{t}^{2}=\phi_{t}^{2}\nu\left(\nu-2\right),\quad\text{for }\nu>2
\]

The conditional density of $Y_{t}\rvert\mathcal{F}_{t-1}$ is given
by
\[
p\left(y_{t}\rvert\mathcal{F}_{t-1}\right)=\frac{\Gamma\left(\frac{\nu+1}{2}\right)}{\sqrt{\pi\nu}\Gamma\left(\frac{\nu}{2}\right)\phi_{t}}\left[1+\frac{y_{t}^{2}}{\nu\phi_{t}^{2}}\right]^{-\frac{\nu+1}{2}}
\]

Derive a GAS model with identity scaling $d=0$ for the scale parameter
$\phi_{t}$. Use an exponential link function to ensure $\phi_{t}>0$
for all $t$.

.

Taking the log of
\begin{align*}
p\left(y_{t}\rvert\mathcal{F}_{t-1}\right) & =\frac{\Gamma\left(\frac{\nu+1}{2}\right)}{\sqrt{\pi\nu}\Gamma\left(\frac{\nu}{2}\right)\phi_{t}}\left[1+\frac{y_{t}^{2}}{\nu\phi_{t}^{2}}\right]^{-\frac{\nu+1}{2}}\\
\ln\left[p\left(y_{t}\rvert\mathcal{F}_{t-1}\right)\right] & =\ln\left[\frac{\Gamma\left(\frac{\nu+1}{2}\right)}{\sqrt{\pi\nu}\Gamma\left(\frac{\nu}{2}\right)\phi_{t}}\left[1+\frac{y_{t}^{2}}{\nu\phi_{t}^{2}}\right]^{-\frac{\nu+1}{2}}\right]
\end{align*}

We know this log density is proportional to. This is due to the fact
that the $\Gamma$, $\nu$ and $\pi$ can be decomposed into constants.
Thus they will not have an effect on the functional form but only
on the level of the process. 
\[
\ln\left[p\left(y_{t}\rvert\mathcal{F}_{t-1}\right)\right]\propto-\ln\left[\phi_{t}\right]-\frac{\nu+1}{2}\ln\left[1+\frac{y_{t}^{2}}{\nu\phi_{t}^{2}}\right]
\]

The score with respect to $\phi_{t}$ is
\begin{align*}
\ln\left[p\left(y_{t}\rvert\mathcal{F}_{t-1}\right)\right] & =-\ln\left[\phi_{t}\right]-\frac{\nu+1}{2}\ln\left[1+\frac{y_{t}^{2}}{\nu\phi_{t}^{2}}\right]\\
\nabla_{t}\equiv\frac{\partial\nabla_{t}}{\partial\phi_{t}} & =-\ln\left[\phi_{t}\right]-\frac{\nu+1}{2}\ln\left[1+\frac{y_{t}^{2}}{\nu\phi_{t}^{2}}\right]\\
\nabla_{t} & =-\frac{1}{\phi_{t}}-\underbrace{\frac{\nu+1}{2}\frac{1}{1+\frac{y_{t}^{2}}{\nu\phi_{t}^{2}}}\cdot\left(-\frac{2y_{t}^{2}}{\nu\phi_{t}^{3}}\right)}_{\text{chain rule}}\\
\nabla_{t} & =-\frac{1}{\phi_{t}}+\frac{\nu+1}{\cancel{2}\left(1+\frac{y_{t}^{2}}{\nu\phi_{t}^{2}}\right)}\cdot\frac{\cancel{2}y_{t}^{2}}{\nu\phi_{t}^{3}}\\
\nabla_{t} & =-\frac{1}{\phi_{t}}+\frac{\left(\nu+1\right)\frac{y_{t}^{2}}{\nu\phi_{t}^{3}}}{1+\frac{y_{t}^{2}}{\nu\phi_{t}^{2}}}\\
\nabla_{t} & =-\frac{1}{\phi_{t}}+\frac{\left(\nu+1\right)y_{t}^{2}}{\left(1+\frac{y_{t}^{2}}{\nu\phi_{t}^{2}}\right)\nu\phi_{t}^{3}}\\
\nabla_{t} & =-\frac{1}{\phi_{t}}+\frac{\left(\nu+1\right)y_{t}^{2}}{\left(\nu\phi_{t}^{3}+\frac{\nu\phi_{t}^{3}y_{t}^{2}}{\nu\phi_{t}^{2}}\right)}\\
\nabla_{t} & =-\frac{1}{\phi_{t}}+\frac{\left(\nu+1\right)y_{t}^{2}}{\left(\nu\phi_{t}^{3}+\phi_{t}y_{t}^{2}\right)}
\end{align*}

Replicating from Leo's solution
\begin{align*}
\underset{\text{multiplying }\frac{2}{2}}{\Longrightarrow}\quad\nabla_{t} & =-\frac{1}{\phi_{t}}+\frac{\left(\nu+1\right)}{2}\frac{2y_{t}^{2}}{\left(\nu\phi_{t}^{3}+\phi_{t}y_{t}^{2}\right)}\\
\nabla_{t} & =-\frac{1}{\phi_{t}}+\frac{\left(\nu+1\right)}{2}\frac{2\frac{y_{t}^{2}}{\phi_{t}^{2}}}{\phi_{t}\left(\nu+\frac{y_{t}^{2}}{\phi_{t}^{2}}\right)}\\
\underset{\text{defining}}{\Longrightarrow}\quad z_{t} & \equiv\frac{y_{t}}{\phi_{t}}\\
\nabla_{t} & =-\frac{1}{\phi_{t}}+\frac{\left(\nu+1\right)}{2}\frac{2z_{t}^{2}}{\phi_{t}\left(\nu+z_{t}^{2}\right)}\\
\nabla_{t} & =-\frac{1}{\phi_{t}}+\frac{\left(\nu+1\right)z_{t}^{2}}{\phi_{t}\left(\nu+z_{t}^{2}\right)}\\
\nabla_{t} & =\frac{\left(\nu+1\right)z_{t}^{2}}{\phi_{t}\left(\nu+z_{t}^{2}\right)}-\frac{1}{\phi_{t}}\\
\underset{\text{factorize}}{\Longrightarrow}\quad\nabla_{t} & =\frac{1}{\phi_{t}}\left[\frac{\left(\nu+1\right)z_{t}^{2}}{\nu+z_{t}^{2}}-1\right]
\end{align*}

We remember that we are to derive a GAS model with identity scaling
$d=0$. Thus we know that
\[
d=0\quad\rightarrow\quad\widetilde{u}_{t}=\frac{\partial\phi_{t}}{\partial\widetilde{\phi}_{t}}\nabla_{t}
\]

The GAS model for $\phi_{t}$ is
\begin{align*}
\phi_{t} & =\lambda\left(\widetilde{\phi}_{t}\right)\quad\underset{\text{exponential link function}}{\Longrightarrow}\quad\phi_{t}=\exp\left(\widetilde{\phi}_{t}\right)\\
\widetilde{\phi}_{t} & =\omega+\alpha\widetilde{u}_{t-1}+\beta\widetilde{\phi}_{t-1}
\end{align*}

Then deriving
\begin{align*}
\widetilde{u}_{t} & =\frac{\partial\phi_{t}}{\partial\widetilde{\phi}_{t}}\nabla_{t}\\
\widetilde{u}_{t} & =\frac{\partial}{\partial\widetilde{\phi}_{t}}\left(\exp\left(\widetilde{\phi}_{t}\right)\right)\frac{1}{\phi_{t}}\left[\frac{\left(\nu+1\right)z_{t}^{2}}{\nu+z_{t}^{2}}-1\right]\\
\widetilde{u}_{t} & =\underbrace{\exp\left(\widetilde{\phi}_{t}\right)}_{=\phi_{t}}\frac{1}{\phi_{t}}\left[\frac{\left(\nu+1\right)z_{t}^{2}}{\nu+z_{t}^{2}}-1\right]\\
\widetilde{u}_{t} & =\frac{\left(\nu+1\right)z_{t}^{2}}{\nu+z_{t}^{2}}-1
\end{align*}

Now we can write up the process and we have the model
\[
\widetilde{\phi}_{t}=\omega+\alpha\left[\frac{\left(\nu+1\right)z_{t-1}^{2}}{\nu+z_{t-1}^{2}}-1\right]+\beta\widetilde{\phi}_{t-1}\quad\square
\]


\section{Problem Set 5 - Problem 2)}

\subsection{Derive unconditional mean to initialize $\phi$}

We keep the process for $\widetilde{\phi}_{t}$ in mind
\[
\widetilde{\phi}_{t}=\omega+\alpha\underbrace{\left[\frac{\left(\nu+1\right)z_{t-1}^{2}}{\nu+z_{t-1}^{2}}-1\right]}_{\tilde{u}_{t-1}}+\beta\widetilde{\phi}_{t-1}
\]

Using recursive substitution and unfolding the process of $\widetilde{\phi}_{t}$
we obtain: 
\begin{align*}
\widetilde{\phi}_{t} & =\frac{\omega}{1-\beta}+\alpha\sum_{s=0}^{\infty}\beta^{s}u_{t-s-1}\\
\mathbb{E}\left[\widetilde{\phi}_{t}\right] & =\mathbb{E}\left[\frac{\omega}{1-\beta}+\alpha\sum_{s=0}^{\infty}\beta^{s}u_{t-s-1}\right]\\
\mathbb{E}\left[\widetilde{\phi}_{t}\right] & =\frac{\omega}{1-\beta}+\underbrace{\mathbb{E}\left[\alpha\sum_{s=0}^{\infty}\beta^{s}u_{t-s-1}\right]}_{=0,\text{ dont know why...}}\\
\mathbb{E}\left[\widetilde{\phi}_{t}\right] & =\frac{\omega}{1-\beta}
\end{align*}

This is the \textbf{unconditional mean}. We know that we have an exponential
link function, whereas 
\[
\phi_{t}=\exp\left(\widetilde{\phi}_{t}\right)
\]

Now we can use these values to initialize our GAS log-likelihood function.


\subsection{Derive starting parameters}

A way of initializing $\theta=\left(\omega,\ \alpha,\ \beta,\ \nu\right)$

We can compute: 
\begin{align*}
\hat{\sigma}^{2} & =Var\left(y_{t}\right)\\
 & =\frac{1}{T}\sum_{t}y_{t}^{2}
\end{align*}

\begin{align*}
\sigma_{t}^{2} & =\frac{\phi_{t}^{2}\nu}{\nu-2}\\
E\left(\sigma_{t}^{2}\right) & =\frac{\nu}{\nu-2}E\left(\exp\left(2\hat{\phi}_{t}\right)\right)
\end{align*}

This expectation can be derived analytically, but this is very technical.
Therefore we use an approximation. 
\begin{align*}
E\left(\sigma_{t}^{2}\right) & =\frac{\nu}{\nu-2}E\left(\exp\left(2\hat{\phi}_{t}\right)\right)\\
 & \approx\frac{\nu}{\nu-2}\exp\left(2E\left(\hat{\phi}_{t}\right)\right)\\
 & =\frac{\nu}{\nu-2}\exp\left(\frac{2\omega}{1-\beta}\right)
\end{align*}

We initialize $\omega$: 
\begin{align*}
\sigma^{2}\frac{\left(\nu-2\right)}{\nu} & =\exp\left(\frac{2\omega}{1-\beta}\right)\\
\ln\left(\sigma^{2}\frac{\left(\nu-2\right)}{\nu}\right)\frac{1}{2}\left(1-\beta\right) & =\omega
\end{align*}

(Note that $\sigma^{2}$ is here the variance of $Y$)

So 
\[
\omega^{\text{initialized}}=\ln\left(\hat{\sigma}^{2}\frac{\left(\nu-2\right)}{\nu}\right)\frac{1}{2}\left(1-\beta\right)
\]


\end{document}
