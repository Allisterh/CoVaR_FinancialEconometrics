\documentclass{article}

%----------------------------------------------------------------------------------------
%	FONTS AND ENCODINGS
%----------------------------------------------------------------------------------------
\usepackage[utf8]{inputenc} % Required for inputting international characters
\usepackage[T1]{fontenc} % Output font encoding for international characters
\usepackage[english]{babel} % language pack..
\usepackage{amsmath} % math symbols
\usepackage{lastpage}
\usepackage{amssymb} % math symbols
\usepackage{cmbright} % font
\usepackage{pifont}
\usepackage{graphicx,float} % for pictures etc.
\usepackage{subcaption}
\usepackage{tcolorbox}
\usepackage{ulem}
\usepackage{cancel}
\usepackage{array}
\usepackage{xcolor}
\usepackage{slashed}
\usepackage{multirow}
\usepackage{wasysym}
\usepackage{minted}
\usepackage[hidelinks]{hyperref}
\usepackage{subfiles}

%----------------------------------------------------------------------------------------
%	PAGE LAYOUT
%----------------------------------------------------------------------------------------
\setlength{\parindent}{0bp} % No paragraph indentation
\usepackage{fancyhdr} % Headings/footers

\usepackage{geometry} % Required for adjusting page dimensions and margins
\geometry{
	paper=a4paper, % Paper size, change to letterpaper for US letter size
	top=2cm, % Top margin
	bottom=2cm, % Bottom margin
	left=3cm, % Left margin
	right=3cm, % Right margin
	inner=2cm,
	outer=2cm,
	headheight=2cm, % Header height
	footskip=1cm, % Space from the bottom margin to the baseline of the footer
	headsep=0.75cm, % Space from the top margin to the baseline of the header
}

\fancyhead{} % Clear all headers
\fancyfoot{} % Clear all footers
\fancyfoot[R]{\footnotesize \thepage\ / \pageref{LastPage}}
\fancyhead[L]{} %Left
\fancyhead[C]{} %Center
\fancyhead[R]{} %Right

\renewcommand{\headrulewidth}{0pt} % Remove header rule
\renewcommand{\footrulewidth}{0pt} % Remove footer rule
\pagestyle{fancy} % Set page style to "fancy"


%----------------------------------------------------------------------------------------
%	DOCUMENT
%----------------------------------------------------------------------------------------

\begin{document}

\section{Problem Set 4 - Problem 1, point 1)}

\subsection{Derivation of density for weights}
In SV model:
$$
y_{t}=\exp \left(\frac{\alpha_{t}}{2}\right) \eta
$$
Factorize joint density:
$$
p\left(Y_{1: t} \mid \alpha_{1: t}\right)=\prod_{s=1}^{t} p\left(y_{t} \mid \alpha_{t}\right)
$$
$\alpha_{t}$ is only dependent on the previosly value of $\alpha_{t-1}$.
The $y$ 's are only dependent through the values of $\alpha$.
$$
p\left(Y_{1: t} \mid \alpha_{1: t}\right)=\prod_{s=1}^{t} \exp \left(-\frac{\alpha_{t}}{2}\right) p\left(y_{t} \exp \left(\frac{\alpha_{t}}{2}\right)\right)
$$
When coding this in $\mathrm{R}$, then it is a matter of foreloops, evaluating normal densities and taking their products. Slide 12:
Choice of function $x_{t}$ :
$$
x_{t}\left(\alpha_{1: t}\right)=\exp \left(\frac{\alpha_{t}}{2}\right)
$$


\section{Problem Set 4 - Problem 1, point 3)}

\subsection{Derive unconditional mean:}

\begin{align*}
\alpha_{t} & =\omega+\phi\alpha_{t-1}+\tau\eta_{t}\\
\underset{\text{Unconditional mean}}{\Longrightarrow}\quad\mathbb{E}\left[\alpha_{t}\right] & =\mathbb{E}\left[\omega+\phi\alpha_{t-1}+\tau\eta_{t}\right]\\
\mathbb{E}\left[\alpha\right] & =\mathbb{E}\left[\omega+\phi\alpha+\tau\eta_{t}\right]\\
\mathbb{E}\left[\alpha\right] & =\mathbb{E}\left[\omega\right]+\mathbb{E}\left[\phi\alpha\right]+\mathbb{E}\left[\tau\eta_{t}\right]\\
\mathbb{E}\left[\alpha\right] & =\mathbb{E}\left[\omega\right]+\mathbb{E}\left[\phi\alpha\right]+\underbrace{\mathbb{E}\left[\tau\eta_{t}\right]}_{=0}\\
\mathbb{E}\left[\alpha\right] & =\omega+\phi\mathbb{E}\left[\alpha\right]\\
\mathbb{E}\left[\alpha\right]-\phi\mathbb{E}\left[\alpha\right] & =\omega\\
\mathbb{E}\left[\alpha\right]\left(1-\phi\right) & =\omega\\
\mathbb{E}\left[\alpha\right] & =\frac{\omega}{\left(1-\phi\right)}
\end{align*}


\subsection{Derive unconditional variance:}

\begin{align*}
\alpha_{t} & =\omega+\phi\alpha_{t-1}+\tau\eta_{t}\\
V\left[\alpha_{t}\right] & =V\left[\omega+\phi\alpha_{t-1}+\tau\eta_{t}\right]\\
\underset{\text{Unconditional variance}}{\Longrightarrow}\quad V\left[\alpha\right] & =V\left[\omega+\phi\alpha+\tau\eta_{t}\right]\\
V\left[\alpha\right] & =V\left[\omega\right]+\phi^{2}V\left[\alpha\right]+\tau^{2}V\left[\eta_{t}\right]\\
V\left[\alpha\right] & =V\left[\omega\right]+\phi^{2}V\left[\alpha\right]+\tau^{2}\underbrace{V\left[\eta_{t}\right]}_{=1\text{, by assump.}}\\
V\left[\alpha\right]-\phi^{2}V\left[\alpha\right] & =V\left[\omega\right]+\tau^{2}\\
V\left[\alpha\right]\left(1-\phi^{2}\right) & =\underbrace{V\left[\omega\right]}_{=0,\text{ constant}}+\tau^{2}\\
V\left[\alpha\right] & =\frac{\tau^{2}}{\left(1-\phi^{2}\right)}
\end{align*}

\end{document}
