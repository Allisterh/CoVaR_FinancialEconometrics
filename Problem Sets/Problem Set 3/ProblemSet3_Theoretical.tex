\documentclass{article}

%----------------------------------------------------------------------------------------
%	FONTS AND ENCODINGS
%----------------------------------------------------------------------------------------
\usepackage[utf8]{inputenc} % Required for inputting international characters
\usepackage[T1]{fontenc} % Output font encoding for international characters
\usepackage[english]{babel} % language pack..
\usepackage{amsmath} % math symbols
\usepackage{lastpage}
\usepackage{amssymb} % math symbols
\usepackage{cmbright} % font
\usepackage{pifont}
\usepackage{graphicx,float} % for pictures etc.
\usepackage{subcaption}
\usepackage{tcolorbox}
\usepackage{ulem}
\usepackage{cancel}
\usepackage{array}
\usepackage{xcolor}
\usepackage{slashed}
\usepackage{multirow}
\usepackage{wasysym}
\usepackage{minted}
\usepackage[hidelinks]{hyperref}
\usepackage{subfiles}

%----------------------------------------------------------------------------------------
%	PAGE LAYOUT
%----------------------------------------------------------------------------------------
\setlength{\parindent}{0bp} % No paragraph indentation
\usepackage{fancyhdr} % Headings/footers

\usepackage{geometry} % Required for adjusting page dimensions and margins
\geometry{
	paper=a4paper, % Paper size, change to letterpaper for US letter size
	top=2cm, % Top margin
	bottom=2cm, % Bottom margin
	left=3cm, % Left margin
	right=3cm, % Right margin
	inner=2cm,
	outer=2cm,
	headheight=2cm, % Header height
	footskip=1cm, % Space from the bottom margin to the baseline of the footer
	headsep=0.75cm, % Space from the top margin to the baseline of the header
}

\fancyhead{} % Clear all headers
\fancyfoot{} % Clear all footers
\fancyfoot[R]{\footnotesize \thepage\ / \pageref{LastPage}}
\fancyhead[L]{} %Left
\fancyhead[C]{} %Center
\fancyhead[R]{} %Right

\renewcommand{\headrulewidth}{0pt} % Remove header rule
\renewcommand{\footrulewidth}{0pt} % Remove footer rule
\pagestyle{fancy} % Set page style to "fancy"


%----------------------------------------------------------------------------------------
%	DOCUMENT
%----------------------------------------------------------------------------------------

\begin{document}


\section{Problem Set 3 - Problem 1}

Consider the SV model reported in slide 31 of Lecture 7 and the SV
model reported in slide 18 of Lecture 6 . Note that the two models
are parameterized in a different way. In the one of Lecture 7 the
$\log$ volatility follows a zero mean autoregression and you have
a parameter $\sigma$ in the measurement equation. In the one of Lecture
6 you have that the volatility follows a first order autoregression
with mean $\omega/(1-\phi)$. Find the mapping between the two parameterizations,
i.e. find a way to represent the model in Lecture 6 as the model in
Lecture 7 , and viceversa.

\bigskip

\textbf{A) From slide 31, lecture 7}

\begin{align*}
y_{t} & =\sigma\exp\left(\frac{w_{t}}{2}\right)z_{t}\\
w_{t} & =\rho w_{t-1}+\eta_{t},\quad\eta_{t}\overset{iid}{\sim}\mathcal{N}\left(0,\sigma_{\eta}^{2}\right)
\end{align*}

\textbf{B) From slide 18, lecture 6 }

\begin{align*}
y_{t} & =\exp\left(\frac{w_{t}}{2}\right)u_{t}\\
w_{t} & =\omega+\phi w_{t-1}+\eta_{t},\quad\eta_{t}\overset{iid}{\sim}\mathcal{N}\left(0,\sigma_{\eta}^{2}\right)
\end{align*}

Let $\theta_{A}=\left(\sigma,\rho,\sigma_{\eta}^{2}\right)^{\prime}$
and $\theta_{B}=\left(\omega,\phi,\sigma_{\eta}^{2}\right)^{\prime}$


\subsection{From (A) to (B)}

\begin{align*}
y_{t} & =\sigma\exp\left(\frac{w_{t}}{2}\right)z_{t}\\
\underset{\ln\left(x^{2}\right)=2\ln\left(x\right),\;\exp\left[\ln\left(x\right)\right]=x}{\Longrightarrow}\quad y_{t} & =\exp\left[\frac{1}{2}\ln\left\{ \sigma^{2}\right\} \right]\exp\left(\frac{w_{t}}{2}\right)z_{t}\\
y_{t} & =e^{\frac{1}{2}\ln\left\{ \sigma^{2}\right\} }e^{\frac{w_{t}}{2}}z_{t}\\
y_{t} & =e^{\frac{1}{2}\ln\left\{ \sigma^{2}\right\} +\frac{w_{t}}{2}}z_{t}\\
y_{t} & =\exp\left[\frac{1}{2}\ln\left\{ \sigma^{2}\right\} +\frac{w_{t}}{2}\right]z_{t}\\
y_{t} & =\exp\left[\frac{1}{2}\underbrace{\left(\ln\left\{ \sigma^{2}\right\} +w_{t}\right)}_{w_{t}^{\star}}\right]z_{t}\\
y_{t} & =\exp\left(\frac{w_{t}^{\star}}{2}\right)z_{t}
\end{align*}

Note that $w_{t}^{\star}$ is the solution of an $AR\left(1\right)$
process of the kind
\[
w_{t}^{\star}=\ln\left(\sigma^{2}\right)\left(1-\rho\right)+\rho w_{t-1}^{\star}+\eta_{t}
\]

Thus we get the mappings
\[
\theta_{A}\rightarrow\theta_{B}=\begin{cases}
\omega & =\ln\left(\sigma^{2}\right)\left(1-\rho\right)\\
\phi & =\rho\\
\sigma_{\eta}^{2} & =\sigma_{\eta}^{2}
\end{cases}
\]

\subsection{From (B) to (A)}

\textbf{TODO: make this derivation thoroughly.}

\bigskip

Thus we get the mappings
\[
\theta_{B}\rightarrow\theta_{A}=\begin{cases}
\sigma & =\exp\left\{ \frac{\omega}{2\left(1-\rho\right)}\right\} \\
\phi & =\rho\\
\sigma_{\eta}^{2} & =\sigma_{\eta}^{2}
\end{cases}
\]

Note that, 
\begin{align*}
\sigma & =\exp\left\{ \frac{\omega}{2\left(1-\rho\right)}\right\} \\
\ln\left(\sigma\right) & =\frac{\omega}{2\left(1-\rho\right)}\\
\omega & =\left(1-\rho\right)\cdot\ln\left(\sigma^{2}\right)
\end{align*}

i.e. the mapping is bijective.


\end{document}
