%% LyX 2.3.6.2 created this file.  For more info, see http://www.lyx.org/.
%% Do not edit unless you really know what you are doing.
\documentclass[english]{article}
\usepackage[latin9]{inputenc}
\usepackage{float}
\usepackage{amsmath}
\usepackage{amssymb}
\usepackage{graphicx}

\makeatletter
%%%%%%%%%%%%%%%%%%%%%%%%%%%%%% User specified LaTeX commands.
\usepackage{booktabs} 
\usepackage{longtable}

\@ifundefined{showcaptionsetup}{}{%
 \PassOptionsToPackage{caption=false}{subfig}}
\usepackage{subfig}
\makeatother

\usepackage{babel}
\begin{document}

\section*{Exam 2019}

\section{Methodology}

Let $Y_{t}$be the VIX at time $t$ adn consider the following model:

\[
\left.Y_{t}\right|F_{t-1}\sim\mathcal{G}\left(\mu_{t},a\right)
\]

Where $\mathcal{G}\left(\mu_{t},a\right)$ is the Gamma distribution
with mean $\mu_{t}>0$ and scale $a>0$ with probability density function
given by: 
\[
p\left(y_{t}\mid\mathcal{F}_{t-1}\right)=\frac{1}{\Gamma(a)}a^{a}y_{t}^{a-1}\mu_{t}^{-a}exp\left(-a\frac{y_{t}}{\mu_{t}}\right)
\]

Here we implement the parameterization of the Gamma distrubution used
by Engle and Gallo for which $E\left[\left.Y_{t}\right|F_{t-1}\right]=\mu_{t}$
and $Var\left[\left.Y_{t}\right|F_{t-1}\right]=\frac{\mu_{t}^{2}}{a}$

Form L9S15 we got. GAS implements this filter as or an updating score
as: 
\begin{align*}
\psi_{t} & =\psi\left(y_{1:t-1}\right)\\
 & =\omega+\alpha u_{t-1}+\beta\psi_{t-1}
\end{align*}

Where $\psi_{t}$ is the variable of interest, and the one we want
to filter out. As we have that $d=\frac{1}{2}$, from L9S19, we have
that the choice of $S_{t}=I_{t}^{-1}$ will then become the inverse
square root scaling $\mu_{y}=I_{t}^{-\frac{1}{2}}\nabla_{t}$ such
that $Var\left(u_{t}\right)=1$. Where $I$ is the Fischer information
matrix and further more from L10S15 $I_{t}^{-1}=E_{t-1}\left[\nabla_{t}^{2}\right]^{-1}$.
$\nabla_{t}$ �s the unscaled score of the conditional distribution
$\nabla_{t}=\frac{\text{\ensuremath{\partial}}logp\left(\left.y_{t}\right|\mathbf{y}_{1:t-1},\varphi_{t},a\right)}{\text{\ensuremath{\partial}}\mu_{t}}$.
The approach is to use the score in order to 'force' the direction
of the updating step, which in our case has been scaled. We will introduce
a link function to accomodate the specifications of $\mu$, i.e. that
it is always positive, and from L9S24, we have that when $d\neq1$
we employ an exponenttial link function

\begin{align}
\mu_{t}= & exp\left(\tilde{\mu}_{t}\right)\\
\tilde{\mu}_{t}= & \omega+\alpha\tilde{u}_{t-1}+\beta\tilde{\mu}_{t-1}
\end{align}

As we have $d=\frac{1}{2},$we have that $u_{t}=\tilde{u_{t}}=I_{t}^{^{-\frac{1}{2}}}\nabla_{t}$
i.e $\rightarrow$ We would like the score, which is defined as the
derivative of the log density
\begin{align*}
\nabla_{t} & =\frac{\text{\ensuremath{\partial}}logp\left(\left.y_{t}\right|\mathbf{y}_{1:t-1},\varphi_{t},a\right)}{\text{\ensuremath{\partial}}\mu_{t}}\\
S_{t} & =I_{t}^{-\frac{1}{2}}=E_{t-1}\left[\nabla_{t}^{2}\right]^{-\frac{1}{2}}
\end{align*}

Note, the condition that $d=\frac{1}{2}$ imply that $u=\tilde{u}$
arises from the fact that our reparameterized model require a reparameterized
$u_{t}$, i.e.:

\begin{align*}
u & =\tilde{u}=\tilde{S_{t}}\tilde{\nabla_{t}}\\
 & =\tilde{I}_{t}^{-\frac{1}{2}}\tilde{\nabla}_{t}\\
 & =\tilde{I}_{t}^{-\frac{1}{2}}\frac{\text{\ensuremath{\partial}}logp\left(\left.y_{t}\right|\mathbf{y}_{1:t-1},\varphi_{t},a\right)}{\text{\ensuremath{\partial}}\tilde{\mu_{t}}}\\
 & =E_{t-1}\left[\tilde{\nabla}_{t}^{2}\right]^{-\frac{1}{2}}\frac{\text{\ensuremath{\partial}}logp\left(\left.y_{t}\right|\mathbf{y}_{1:t-1},\varphi_{t},a\right)}{\text{\ensuremath{\partial}}\tilde{\mu}_{t}}\\
 & =E_{t-1}\left[\left(\frac{\text{\ensuremath{\partial}}logp\left(\left.y_{t}\right|\mathbf{y}_{1:t-1},\varphi_{t},a\right)}{\text{\ensuremath{\partial}}\tilde{\mu_{t}}}\right)^{2}\right]^{-\frac{1}{2}}\frac{\text{\ensuremath{\partial}}logp\left(\left.y_{t}\right|\mathbf{y}_{1:t-1},\varphi_{t},a\right)}{\text{\ensuremath{\partial}}\tilde{\mu_{t}}}\\
 & =E_{t-1}\left[\left(\frac{\text{\ensuremath{\partial}}logp\left(\left.y_{t}\right|\mathbf{y}_{1:t-1},\varphi_{t},a\right)}{\text{\ensuremath{\partial}}\mu_{t}}*\frac{\text{\ensuremath{\partial}}\mu_{t}}{\text{\ensuremath{\partial}}\tilde{\mu_{t}}}\right)^{2}\right]^{-\frac{1}{2}}\frac{\text{\ensuremath{\partial}}logp\left(\left.y_{t}\right|\mathbf{y}_{1:t-1},\varphi_{t},a\right)}{\text{\ensuremath{\partial}}\mu_{t}}\frac{\text{\ensuremath{\partial}}\mu_{t}}{\text{\ensuremath{\partial}}\tilde{\mu_{t}}}\\
 & =E_{t-1}\left[\left(\nabla_{t}*\frac{\text{\ensuremath{\partial}}\mu_{t}}{\text{\ensuremath{\partial}}\tilde{\mu_{t}}}\right)^{2}\right]^{-\frac{1}{2}}\nabla_{t}\frac{\text{\ensuremath{\partial}}\mu_{t}}{\text{\ensuremath{\partial}}\tilde{\mu_{t}}}\\
 & =E_{t-1}\left[\nabla_{t}^{2}*\left(\frac{\text{\ensuremath{\partial}}\mu_{t}}{\text{\ensuremath{\partial}}\tilde{\mu_{t}}}\right)^{2}\right]^{-\frac{1}{2}}\nabla_{t}\frac{\text{\ensuremath{\partial}}\mu_{t}}{\text{\ensuremath{\partial}}\tilde{\mu_{t}}}\\
 & =E_{t-1}\left[\nabla_{t}^{2}\right]^{-\frac{1}{2}}\left(\frac{\text{\ensuremath{\partial}}\mu_{t}}{\text{\ensuremath{\partial}}\tilde{\mu_{t}}}\right)^{2*-\frac{1}{2}}\nabla_{t}\frac{\text{\ensuremath{\partial}}\mu_{t}}{\text{\ensuremath{\partial}}\tilde{\mu_{t}}}\\
 & =E_{t-1}\left[\nabla_{t}^{2}\right]^{-\frac{1}{2}}\left(\frac{\text{\ensuremath{\partial}}\mu_{t}}{\text{\ensuremath{\partial}}\tilde{\mu_{t}}}\right)^{-\frac{1}{1}}\nabla_{t}\frac{\text{\ensuremath{\partial}}\mu_{t}}{\text{\ensuremath{\partial}}\tilde{\mu_{t}}}\\
 & =E_{t-1}\left[\nabla_{t}^{2}\right]^{-\frac{1}{2}}\nabla_{t}\frac{\text{\ensuremath{\partial}}\mu_{t}}{\text{\ensuremath{\partial}}\tilde{\mu_{t}}}\frac{\text{\ensuremath{\partial}}\tilde{\mu_{t}}}{\text{\ensuremath{\partial}}\mu_{t}}\\
 & =E_{t-1}\left[\nabla_{t}^{2}\right]^{-\frac{1}{2}}\nabla_{t}\\
u=\tilde{u} & =S_{t}\nabla_{t}
\end{align*}

This is the forcing variable. Note, I have used the fact that $\tilde{\mu}_{t}$
and $\mu_{t}$ is adopted to the filtration at time $t-1$, hence
one may just take it out of the expectation sign.\\
This is for the more general case. For the model of the question:

Taking logs to the probability density function of the model we get.
\begin{align*}
log\left(p\left(y_{t}\mid\mathcal{F}_{t-1}\right)\right) & =log\left(\frac{1}{\Gamma(a)}a^{a}y_{t}^{a-1}\mu_{t}^{-a}exp\left(-a\frac{y_{t}}{\mu_{t}}\right)\right)\\
 & =log\left(\frac{1}{\Gamma(a)}\right)+log\left(a^{a}\right)+log\left(y_{t}^{a-1}\right)+log\left(\mu_{t}^{-a}\right)+log\left(exp\left(-a\frac{y_{t}}{\mu_{t}}\right)\right)\\
 & =-log\left(\Gamma\left(a\right)\right)+alog\left(a\right)+\left(a-1\right)log\left(y_{t}\right)+\left(-a\right)log\left(\mu_{t}\right)-a\frac{y_{t}}{\mu_{t}}
\end{align*}

Taking the score of this by: $\nabla_{t}=\frac{\text{\ensuremath{\partial}}logp\left(\left.y_{t}\right|\mathbf{y}_{1:t-1},\varphi_{t},a\right)}{\text{\ensuremath{\partial}}\mu_{t}}$
\begin{align*}
\nabla_{t} & =\frac{\text{\ensuremath{\partial}}}{\text{\ensuremath{\partial}}\mu_{t}}\left(logp\left(\left.y_{t}\right|\mathbf{y}_{1:t-1},\varphi_{t},a\right)\right)\\
 & =\frac{\text{\ensuremath{\partial}}}{\text{\ensuremath{\partial}}\mu_{t}}\left(-log\left(\Gamma\left(a\right)\right)+alog\left(a\right)+\left(a-1\right)log\left(y_{t}\right)+\left(-a\right)log\left(\mu_{t}\right)-a\frac{y_{t}}{\mu_{t}}\right)\\
 & =-a\frac{1}{\mu_{t}}+a\frac{y_{t}}{\mu_{t}^{2}}
\end{align*}

This being unscaled score of the model, for this question, we have
scaled the model by $\nabla_{t}^{2}$ and thereby

\begin{align*}
\nabla_{t}^{2} & =\left(\frac{ay_{t}}{\mu_{t}^{2}}-\frac{a}{\mu_{t}}\right)^{2}\\
 & =\left(\frac{ay_{t}}{\mu_{t}^{2}}\right)^{2}-\left(\frac{a}{\mu_{t}}\right)^{2}-2*\left(\frac{ay_{t}}{\mu_{t}^{2}}\frac{a}{\mu_{t}}\right)\\
 & =\left(\frac{ay_{t}}{\mu_{t}^{2}}\right)^{2}-\left(\frac{a}{\mu_{t}}\right)^{2}-\frac{2a^{2}y_{t}}{\mu_{t}^{3}}\\
 & =\frac{a^{2}y_{t}^{2}}{\mu_{t}^{4}}+\frac{a^{2}}{\mu_{t}^{2}}-\frac{2a^{2}y_{t}}{\mu_{t}^{3}}\\
 & =a^{2}\left(\frac{y_{t}^{2}}{\mu_{t}^{4}}+\frac{1}{\mu_{t}^{2}}-\frac{2y_{t}}{\mu_{t}^{3}}\right)
\end{align*}

As on L9S18, we have that $Var\left(\nabla_{t}\right)=E_{t-1}\left[\nabla_{t}^{2}\right]=I_{t}$,
and where we use what is in the question about the $E_{t}\left[\left.Y_{t}\right|F_{t-1}\right]=\mu_{t}$
and $E_{t}\left[\left.Y_{t}^{2}\right|F_{t-1}\right]=\frac{\mu_{t}^{2}\left(1+a\right)}{a}$
\begin{align*}
E_{t-1}\left[\nabla_{t}^{2}\right] & =E_{t-1}\left[a^{2}\left(\frac{y_{t}^{2}}{\mu_{t}^{4}}+\frac{1}{\mu_{t}^{2}}-\frac{2y_{t}}{\mu_{t}^{3}}\right)\right]\\
 & =a^{2}E_{t-1}\left[\frac{y_{t}^{2}}{\mu_{t}^{4}}+\frac{1}{\mu_{t}^{2}}-\frac{2y_{t}}{\mu_{t}^{3}}\right]\\
 & =a^{2}\left(E_{t-1}\left[\frac{y_{t}^{2}}{\mu_{t}^{4}}\right]+E_{t-1}\left[\frac{1}{\mu_{t}^{2}}\right]-E_{t-1}\left[\frac{2y_{t}}{\mu_{t}^{3}}\right]\right)\\
 & =a^{2}\left(\frac{1}{\mu_{t}^{4}}\frac{\mu_{t}^{2}\left(1+a\right)}{a}+\frac{1}{\mu_{t}^{2}}-\frac{2}{\mu_{t}^{3}}\mu_{t}\right)\\
 & =a^{2}\left(\frac{\mu_{t}^{2}\left(1+a\right)}{a\mu_{t}^{4}}-\frac{1}{\mu_{t}^{2}}\right)\\
 & =\left(\frac{a+a^{2}}{\mu_{t}^{2}}-\frac{a^{2}}{\mu_{t}^{2}}\right)\\
 & =\left(\frac{a}{\mu_{t}^{2}}+\frac{a^{2}}{\mu_{t}^{2}}-\frac{a^{2}}{\mu_{t}^{2}}\right)\\
 & =\frac{a}{\mu_{t}^{2}}
\end{align*}

where the fact that $\mu_{t}$ is constant given the information at
time $t-1$ - it is adopted to the filtration at time $t-1$ - hence
we may treat is a constant and just take it out of the expectation
sign.

Next, one need to recall that we need to find $\mathcal{I}^{-\frac{1}{2}}$,
hence we get:
\[
\mathcal{I}=\frac{a}{\mu_{t}^{2}}\rightarrow\mathcal{I}^{-\frac{1}{2}}=\frac{\text{\ensuremath{\mu_{t}}}}{a^{\frac{1}{2}}}
\]

\[
\mathcal{I}^{-\frac{1}{2}}=\frac{\mu_{t}}{\sqrt{a}}.
\]

Now putting the pieces together we get: $\rightarrow$ forccing variable
is then $Eq.$
\begin{align}
u_{t}=\tilde{u}_{t} & =\mathcal{I}^{-\frac{1}{2}}\nabla_{t}\nonumber \\
 & =\frac{\mu_{t}}{\sqrt{a}}*\left(\frac{ay_{t}}{\mu_{t}^{2}}-\frac{a}{\mu_{t}}\right)\nonumber \\
 & =\frac{1}{\sqrt{a}}*\left(\mu_{t}\frac{ay_{t}}{\mu_{t}^{2}}-\mu_{t}\frac{a}{\mu_{t}}\right)\nonumber \\
 & =\frac{1}{a^{\frac{1}{2}}}*\left(\frac{ay_{t}}{\mu_{t}}-a\right)\nonumber \\
 & =a^{-\frac{1}{2}}*a\left(\frac{y_{t}}{\mu_{t}}-1\right)\nonumber \\
 & =a^{\frac{1}{2}}\left(\frac{y_{t}}{\mu_{t}}-1\right)\nonumber \\
u_{t} & =a^{\frac{1}{2}}\left(\frac{y_{t}}{exp\left(\tilde{\mu_{t}}\right)}-1\right)
\end{align}

Where $Eq$. now can be interpretede as the the updating step. We
now have the Gamma-GAS updating step as

\begin{align}
\mu_{t} & =exp\left(\tilde{\mu_{t}}\right)\\
\tilde{\mu}_{t} & =\omega+\alpha\tilde{u}_{t-1}+\beta\tilde{\mu}_{t-1}\\
\tilde{\mu}_{t} & =a^{\frac{1}{2}}\left(\frac{y_{t}}{exp\left(\tilde{\mu_{t}}\right)}-1\right)
\end{align}

From L9S22 we have that when $d=\frac{1}{2}$ $\tilde{\mu}_{t}=\mu_{t}$
and from L9S and when $d=\frac{1}{2}$: Inverse square root scaling
$u_{t}=I_{t}^{-\frac{1}{2}}\nabla_{t}$ such that $Var\left(u_{y}\right)=Var\left(\tilde{u}_{t}\right)=1$
i.e. taking the variance to $\tilde{\mu_{t}}$ should equal 1.

\begin{align*}
Var\left(\tilde{\mu}_{t}\right) & =Var\left(a^{\frac{1}{2}}\left(\frac{y_{t}}{exp\left(\tilde{\mu_{t}}\right)}-1\right)\right)\\
 & =a^{\frac{1}{2}*2}Var\left(\frac{y_{t}}{exp\left(\tilde{\mu_{t}}\right)}-1\right)\\
 & =a*\left(\frac{1}{exp\left(\tilde{\mu}_{t}\right)}\right)^{2}Var\left(y_{t}\right)\\
 & =a*\frac{1}{exp\left(\tilde{\mu}_{t}\right)^{2}}\frac{\mu_{t}^{2}}{a}\\
 & =\frac{a}{\mu_{t}^{2}}\frac{\mu_{t}^{2}}{a}\\
Var\left(\tilde{\mu}_{t}\right) & =1
\end{align*}

where I have used that the variance to a constant is zero and that
$\mu_{t}=exp\left(\tilde{\mu}_{t}\right)$. We may therefore confirm
that we have found the right $u_{t}=\tilde{u}_{t}$, as we have $Var\left(u_{t}\right)=Var\left(\tilde{u}_{t}\right)=1$.

For the loglikelihood of the model we have

In exercise 1.1, the individual log-likelihood contributions has been
derived as:
\begin{equation}
logp\left(\left.y_{t}\right|\mathbf{y}_{1:t-1},\mu_{t},a\right)=-log\left(\Gamma\left(a\right)\right)+alog\left(a\right)+\left(a-1\right)log\left(y_{t}\right)-alog\left(\mu_{t}\right)-a\frac{y_{t}}{\mu_{t}}.
\end{equation}

Adding up, recalling that $\mu_{t}$ is a function of $\left(\omega,\alpha,\beta\right)$,
one obtain for length $T$ the loglikelihood as:
\begin{align}
logL\left(\omega,\alpha,\beta,a\right) & =-\sum_{t=1}^{T}\left[log\left(\Gamma\left(a\right)\right)-alog\left(a\right)-\left(a-1\right)log\left(y_{t}For\right)+alog\left(\mu_{t}\left(\omega,\alpha,\beta\right)\right)+a\frac{y_{t}}{\mu_{t}\left(\omega,\alpha,\beta\right)}\right]\\
 & =T\left(-ln\Game\left(a\right)+aln\left(a\right)\right)+\sum_{t=1}^{T}\left(\left(a-1\right)log\left(y_{t}For\right)+alog\left(\mu_{t}\left(\omega,\alpha,\beta\right)\right)+a\frac{y_{t}}{\mu_{t}\left(\omega,\alpha,\beta\right)}\right)\nonumber 
\end{align}

When considering the GAMMA-GAS model, I have written a few codes;
1) to derive the log-pdf of the gamma distribution, 2) that filter
the means, $\mu_{t}$, and 3) that uses maximum likelihood to estimate
the parameter values in the filter, i.e. $\left(\omega,\alpha,\beta,a\right)$.
When initialising the filter, I have taken the unconditional expectation
to $\text{\ensuremath{\tilde{\mu}}}_{t}$, i.e.:

\begin{align}
E\left[\tilde{\mu}_{t}\right] & =\omega+\alpha\underset{Eq.6}{\underbrace{\sqrt{a}\left(\frac{E\left[y_{t}\right]}{exp\left(\tilde{\mu}_{t}\right)}-1\right)}}+\beta E\left[\mu_{t-1}\right]\nonumber \\
 & =\omega+\alpha\sqrt{a}\left(\frac{\mu_{t}}{\mu_{t}}-1\right)+\beta E\left[\mu_{t-1}\right]\nonumber \\
 & =\omega+\alpha\sqrt{a}\underset{=0}{\underbrace{\left(\frac{\mu_{t}}{\mu_{t}}-1\right)}}+\beta E\left[\mu_{t-1}\right]\nonumber \\
 & =\frac{\omega}{1-\beta}
\end{align}

which will be the starting condition for $\tilde{\mu}_{t}$, i.e.
$\tilde{\mu}_{1}$. Note, we have assumed stationarity in order for
this to hold, i.e. $E\left[\tilde{\mu}_{t}\right]=E\left[\tilde{\mu}_{t-1}\right]$.
Furthermore, from the exercise, the parameter constraints has been
specified for the coefficients $\left(\omega,\alpha,\beta,\alpha\right)$.

\subsection*{Multiplicative Error Model}

After considering the GAMMA-GAS model - constrained and unconstrained
- one moves onto considering the Multiplicative Errror Model by Engle
and Gallo (2006). Notice it is formulated in the following way: 
\begin{align}
\left.Y_{t}\right|\mathcal{F}_{t-1} & \sim\mathcal{G}a\left(\mu_{t},a\right),\nonumber \\
\mu_{t} & =\kappa+\eta y_{t-1}+\phi\mu_{t-1}.
\end{align}

For the estimation of the Multiplicative Error Model, I have created
two functions, 1) that filter the means, $\mu_{t}$, and 2) that uses
maximum likelihood to estimate the parameter values in the filter,
i.e. $\left(\kappa,\eta,\phi,a\right)$. Furthermore it compute the
BIC as well as the loglikelihood.\\

When estimating the filter, I have taken the unconditional expectation
to $\text{\ensuremath{\mu}}_{t}$, i.e.:
\begin{align}
E\left[\mu_{t}\right] & =\kappa+\eta\underset{\mu_{t-1}}{\underbrace{E\left[y_{t-1}\right]}}+\phi\underset{\mu_{t}}{E\underbrace{\left[\mu_{t-1}\right]}}\nonumber \\
 & =\kappa+\eta\mu_{t-1}+\phi E\left[\mu_{t}\right]\nonumber \\
E\left[\mu_{t}\right] & =\kappa+\mu_{t}\left(\eta+\phi\right)\nonumber \\
\kappa & =\mu_{t}\left(1-\eta-\phi\right)\nonumber \\
 & =\frac{\kappa}{1-\eta-\phi},
\end{align}
where I have used two things; 1) $E\left[y_{t}\right]=\mu_{t}$ and
stationarity, i.e. $E\left[\mu_{t}\right]=E\left[\mu_{t-1}\right]$.
Notice that this further imply one condition, which has not been imposed
in the exercise, i.e. that
\begin{align*}
\eta+\phi & <1,
\end{align*}


\section*{Question 2}

\subsection*{Methodology}

\subsection*{Bivariate Gaussian DCC model - model, constraints and factorization
of the likelihood}

We consider a bivariate random vector of the returns of GSPC and DJI
returns. As one notices from the exercise, we assume that the distribution
at time $t$, conditional on the filtration at time $t-1$, is bivarate
gaussian according to:
\begin{equation}
\left.\mathbf{Y}_{t}\right|\mathcal{F}_{t-1}\sim N\left(\mathbf{0},\boldsymbol{\Sigma}_{t}\right).
\end{equation}

Leopoldo further specifies the model as so:

\begin{align*}
\boldsymbol{Y}_{t} & =\left(y_{1,t},...,y_{N,t}\right)^{'}\\
y_{i,t} & =\sigma_{i,t}\epsilon_{t}\\
\sigma_{i,t}^{2} & =\omega+\alpha_{i}y_{i,t-1}^{2}+\beta\sigma_{i,t-1}^{2}
\end{align*}

$\sigma_{i,t}^{2}$ are the marginals GARCH, put into the DCC framework

\[
Y_{y}=\boldsymbol{\Sigma}^{\frac{1}{2}}\epsilon_{t}\rightarrow\boldsymbol{\Sigma}^{\frac{1}{2}}\boldsymbol{\Sigma}^{\frac{1}{2}}=\boldsymbol{D}_{t}\boldsymbol{R}_{t}\boldsymbol{D}_{t}
\]

For when considering the DCC model we turn to L10S24 / L10S30, were
we note that the conditional covariance matriz can be factorized as;
\begin{equation}
\boldsymbol{H}_{t}=\boldsymbol{D}_{t}^{\frac{1}{2}}\boldsymbol{R}_{t}\boldsymbol{D}_{t}^{\frac{1}{2}}
\end{equation}
 where $\boldsymbol{D_{t}}$ is a matrix with the variance on the
diagonol i.e,
\begin{equation}
\boldsymbol{D}_{t}=\left[\begin{array}{ccc}
\sigma_{1,t}^{2} & 0 & 0\\
0 & \sigma_{2,t}^{2} & 0\\
0 & 0 & \sigma_{3,t}^{2}
\end{array}\right]
\end{equation}

and where $\sigma_{i,t}^{2}=var\left(\left.y_{i,t}\right|\mathcal{F}_{t-1}\right)$.
$\boldsymbol{R}_{t}$ is the evaluation matrix with typical element
$\rho_{i;j,t}=cor\left(\left.y_{i,t},y_{j,t}\right|\mathcal{F}_{t-1}\right)$
i.e

\begin{equation}
\boldsymbol{R}_{t}=\left[\begin{array}{ccc}
1 & \rho_{1.2} & \rho_{1.3}\\
\rho_{2,1} & 1 & \rho_{2.3}\\
\rho_{3.1} & \rho_{3.2} & 1
\end{array}\right]
\end{equation}

$Eq.15$ is also what Leopoldo calls $\bar{\boldsymbol{R}}_{\boldsymbol{t}}=cor(\eta_{t})$

From here we turn to L10S31 $\rightarrow$ As specified in the exercise,
in the following, we will allow for the marginal processes or the
$\sigma^{2}$ to follow a $GARCH(1,1)$ process of below form, and
further, one needs to incorporate the assumption about $\mathbf{R}_{t}$:

\begin{equation}
\sigma_{i,j}^{2}=\omega+\alpha y_{i,t-1}^{2}+\beta\sigma_{i.t-1}^{2}
\end{equation}

\begin{equation}
\mathbf{R}_{t}=\tilde{\mathbf{Q}}_{t}^{-\frac{1}{2}}\mathbf{Q}_{t}\tilde{\mathbf{Q}}_{t}^{-\frac{1}{2}}
\end{equation}

Where $\tilde{\boldsymbol{Q}_{t}}$ is a diagonol matrix with typical
elements of $\boldsymbol{Q}_{t}$, which we can write out as on L10S31.

\begin{equation}
\boldsymbol{Q}_{t}=\bar{\boldsymbol{Q}}\left(1-a-b\right)+a\left(\boldsymbol{\eta}_{t-1}\boldsymbol{\eta}_{t-1}^{'}\right)+b\left(\boldsymbol{Q}_{t-1}\right)
\end{equation}

Leopoldo alsp has $\rightarrow$$\boldsymbol{\eta}_{t}=\left(\eta_{1,t},...,\eta_{N,T}\right)^{'},\eta_{i,t}=\frac{y_{i,t}}{\sigma_{i,t}}$

Where $\boldsymbol{\eta}_{t}=\boldsymbol{D}_{t}^{-\frac{1}{2}}\boldsymbol{y}_{t}$
are standarized returns, and $\bar{\boldsymbol{Q}}$ is fixed to the
emperical evaluation of $\text{\ensuremath{\boldsymbol{\eta}_{t}}\ensuremath{\bar{\boldsymbol{Q}_{t}}=\bar{\boldsymbol{R}_{t}}=cor\left(z_{t}\right)} used in the CCC model where the correlation is static.}$Note,
in above formulation, we have the following constraints for our model.
this is to insure both that the variances are positive but $\beta+\alpha\leq1,$
is also for weak stationarity

\[
positivity\ of\ \sigma_{i,t}^{2}=\left\{ \begin{array}{c}
\omega_{i}>0\\
\alpha_{i}>0\\
\beta_{i}>0
\end{array}\right.
\]

\[
weak\ stationarity\ of\ y_{i,t}=\left\{ \alpha_{i}+\beta_{i}<1\right.
\]

\begin{align}
0 & \leq a,b\leq1, & a+b & \leq1,\\
0 & \leq\alpha,\beta<1, & \alpha+\beta & <1.\nonumber 
\end{align}
These constraints do not only cause weak stationarity of the our GARCH(1,1)
process\footnote{Note, the GARCH model has been initialised with its unconditional
variance, i.e. 
\begin{align*}
E\left[\sigma_{i,t}^{2}\right] & =\omega+\alpha\underset{\sigma_{i,t-1}^{2}}{\underbrace{E\left[y_{i,t-1}^{2}\right]}}+\beta\left[\sigma_{i,t-1}^{2}\right]\\
 & =\frac{\omega}{1-\alpha-\beta}.
\end{align*}
}, but also causes the conditional covariance matrix, $\mathbf{Q}_{t}$,
to be positive definite, which in terms imply that $\mathbf{R}_{t}$
is positive definite. As oppose to the CCC model, $\mathbf{Q}_{t}$
allows for $\mathbf{R}_{t}$ to change over time, but a drawback of
the general procedure is that $a$ and $b$ drive the dynamics of
the correlations, which has led to more recent extensions of the DCC
model, which overcomes this feature.\\

Next step is to derive the loglikelihood, hence one consider the density
of $\left.\mathbf{y}_{t}\right|\mathcal{F}_{t-1}$; note $\mu=0$,
hence it reduces to: Simply taking from wiki, and with $\sigma^{2}=\boldsymbol{\Sigma}$
\begin{equation}
p\left(\mathbf{y};0,\boldsymbol{\Sigma}\right)=\frac{1}{2\pi\left|\boldsymbol{\Sigma}_{t}\right|^{\frac{1}{2}}}exp\left\{ -\frac{1}{2}\mathbf{\mathbf{y}_{t}}'\boldsymbol{\Sigma}_{t}^{-1}\mathbf{\mathbf{y}_{t}}\right\} .
\end{equation}

Taking logs yields:

\begin{align}
log\left(p\left(\mathbf{y};0,\boldsymbol{\Sigma}\right)\right) & =log\left(\frac{1}{2\pi\left|\boldsymbol{\Sigma}_{t}\right|^{\frac{1}{2}}}exp\left\{ -\frac{1}{2}\mathbf{\mathbf{y}_{t}}'\boldsymbol{\Sigma}_{t}^{-1}\mathbf{\mathbf{y}_{t}}\right\} \right)\nonumber \\
 & =log\left(2\pi\left|\boldsymbol{\Sigma}_{t}\right|^{\frac{1}{2}}\right)^{-1}exp\left\{ -\frac{1}{2}\mathbf{\mathbf{y}_{t}}'\boldsymbol{\Sigma}_{t}^{-1}\mathbf{\mathbf{y}_{t}}\right\} \nonumber \\
 & =-log\left(2\pi\right)-log\left(\left|\boldsymbol{\Sigma}_{t}\right|^{\frac{1}{2}}\right)exp\left\{ -\frac{1}{2}\mathbf{\mathbf{y}_{t}}'\boldsymbol{\Sigma}_{t}^{-1}\mathbf{\mathbf{y}_{t}}\right\} \nonumber \\
 & =-log\left(2\pi\right)-\frac{1}{2}log\left(\left|\boldsymbol{\Sigma}_{t}\right|\right)-\frac{1}{2}\mathbf{\mathbf{y}_{t}}'\boldsymbol{\Sigma}_{t}^{-1}\mathbf{\mathbf{y}_{t}}
\end{align}

And considering of lenght $T$

\begin{align}
logL_{t} & =\sum_{t=1}^{T}-log\left(2\pi\right)-\frac{1}{2}log\left(\left|\boldsymbol{\Sigma}_{t}\right|\right)-\frac{1}{2}\mathbf{\mathbf{y}_{t}}'\boldsymbol{\Sigma}_{t}^{-1}\mathbf{\mathbf{y}_{t}}\nonumber \\
 & =-\frac{1}{2}\sum_{t=1}^{T}+2log\left(2\pi\right)+log\left(\left|\boldsymbol{\Sigma}_{t}\right|\right)+\mathbf{\mathbf{y}_{t}}'\boldsymbol{\Sigma}_{t}^{-1}\mathbf{\mathbf{y}_{t}}\nonumber \\
 & =-\frac{1}{2}\sum_{t=1}^{T}\left(2log\left(2\pi\right)+log\left(\left|\underset{\boldsymbol{\Sigma}_{t}}{\underbrace{\mathbf{D}_{t}^{\frac{1}{2}}\mathbf{R}_{t}\mathbf{D}_{t}^{\frac{1}{2}}}}\right|\right)+\mathbf{\mathbf{y}}'_{t}\left(\underset{\boldsymbol{\Sigma}_{t}}{\underbrace{\mathbf{D}_{t}^{\frac{1}{2}}\mathbf{R}_{t}\mathbf{D}_{t}^{\frac{1}{2}}}}\right)^{-1}\mathbf{y}_{t}\right)\nonumber \\
 & =-\frac{1}{2}\sum_{t=1}^{T}\left(2log\left(2\pi\right)+log\left(\left|\mathbf{R}_{t}\right|\right)+log\left(\left|\mathbf{D}_{t}\right|\right)+\underset{\boldsymbol{\eta_{t}}}{\underbrace{\mathbf{\mathbf{y}}'_{t}\mathbf{D}_{t}^{-\frac{1}{2}}}}\mathbf{R}_{t}^{-1}\underset{\boldsymbol{\eta_{t}}}{\underbrace{\mathbf{D}_{t}^{-\frac{1}{2}}\mathbf{\mathbf{y}}_{t}}}\right)\nonumber \\
 & =-\frac{1}{2}\sum_{t=1}^{T}\left(2log\left(2\pi\right)+log\left(\left|\mathbf{R}_{t}\right|\right)+log\left(\left|\mathbf{D}_{t}\right|\right)+\boldsymbol{\eta}_{t}'\mathbf{R}_{t}^{-1}\boldsymbol{\eta}_{t}\right)
\end{align}
As introduced in L10S34, we add and substract $\mathbf{Y}_{t}'\mathbf{D}_{t}^{-\frac{1}{2}}\mathbf{D}_{t}^{-\frac{1}{2}}\mathbf{Y}_{t}=\mathbf{\boldsymbol{\eta}}_{t}'\mathbf{\boldsymbol{\eta}}\rightarrow_{t}\mathbf{Y}_{t}'\mathbf{D}_{t}^{-\frac{1}{2}}\mathbf{D}_{t}^{-\frac{1}{2}}\mathbf{Y}_{t}-\mathbf{\boldsymbol{\eta}}_{t}'\mathbf{\boldsymbol{\eta}}$
\begin{align}
logL_{T} & =-\frac{1}{2}\sum_{t=1}^{T}\left(2log\left(2\pi\right)+log\left(\left|\mathbf{D}_{t}\right|\right)+\mathbf{y}_{t}'\mathbf{D}_{t}^{-\frac{1}{2}}\mathbf{D}_{t}^{-\frac{1}{2}}\mathbf{y}_{t}-\mathbf{\boldsymbol{\eta}}_{t}'\mathbf{\boldsymbol{\eta}}_{t}+log\left(\left|\mathbf{R}_{t}\right|\right)+\boldsymbol{\eta}_{t}'\mathbf{R}_{t}^{-1}\boldsymbol{\eta}_{t}\right)\nonumber \\
 & =\underset{logL_{V,T}\left(\theta\right)}{\underbrace{-\frac{1}{2}\sum_{t=1}^{T}\left(2log\left(2\pi\right)+log\left(\left|\mathbf{D}_{t}\right|\right)+\mathbf{y}_{t}'\mathbf{D}_{t}^{-1}\mathbf{y}_{t}\right)}}\;\underset{logL_{C,T}\left(\theta,\phi\right)}{\underbrace{-\frac{1}{2}\sum_{t=1}^{T}\left(\boldsymbol{\eta}_{t}'\mathbf{R}_{t}^{-1}\boldsymbol{\eta}_{t}-\mathbf{\boldsymbol{\eta}}_{t}'\mathbf{\boldsymbol{\eta}}_{t}+log\left(\left|\mathbf{R}_{t}\right|\right)\right)}},
\end{align}

\begin{align}
logL_{V,T}\left(\theta\right)= & -\frac{1}{2}\sum_{t=1}^{T}\left(2log\left(2\pi\right)+log\left(\left|\mathbf{D}_{t}\right|\right)+\mathbf{y}_{t}'\mathbf{D}_{t}^{-1}\mathbf{y}_{t}\right)\\
ogL_{C,T}\left(\theta,\phi\right)= & -\frac{1}{2}\sum_{t=1}^{T}\left(\boldsymbol{\eta}_{t}'\mathbf{R}_{t}^{-1}\boldsymbol{\eta}_{t}-\mathbf{\boldsymbol{\eta}}_{t}'\mathbf{\boldsymbol{\eta}}_{t}+log\left(\left|\mathbf{R}_{t}\right|\right)\right)
\end{align}

The procedure is then to seperately maximizing the likelihoods and
then sum them in order to report the total likelihood of the model.
The procedure is following a two-step approach, where one initially
maximizes the likelihood with regards to the volatility, hence obtain
$\hat{\theta}$, and then one maximizes the likelihood of the correlation
part with regards to $\phi$ in order to obtain $\hat{\phi}$. This
approach is possible due to the nice factorization of the likelihood
functions.\\

For the volatility part of the likelihood, one may write:
\begin{align}
logL_{V,T}\left(\theta\right) & =-\frac{1}{2}\sum_{t=1}^{T}\left(2log\left(2\pi\right)+log\left(\left|\mathbf{D}_{t}\right|\right)+\mathbf{y}_{t}'\mathbf{D}_{t}^{-1}\mathbf{y}_{t}\right)\nonumber \\
 & =-\frac{1}{2}\sum_{t=1}^{T}\sum_{i=1}^{N}\left(2log\left(2\pi\right)+log\left(h_{i,t}\right)+\frac{r_{i,t}^{2}}{h_{i,t}}\right),
\end{align}
which simply imply that we may split the problem into 'severals' i.e.
one may maximize each GARCH likelihood seperately and then sum them
at the end, which is exactly what our function in R will do with the
data on GSPC and DJI.\\

The succes of the procedure crucially depends on consistency of the
first estimate, i.e. $\hat{\theta}$ - if it is consistent, then the
estimate of phi, $\hat{\phi}$, will be consistent assuming continuity
of the loglikelihood function around the true value of $\phi$.\\

As we have specified the DCC model, one may consider the special case
with $\mathbf{R}_{t}=\mathbf{R}$, which is the CCC model. For the
CCC model we set $a=b=0$ in $\boldsymbol{Q}_{t}$, CCC is a special
case of the DCC model.

\subsection*{Constant Conditional Correlation (CCC) Model}

When considering the Constant Conditional Correlation model, we will
make use of most parts of above in regards to the DCC model, but the
crucial point is the definition of $\mathbf{R}$, where the CCC model
make use of a time-invariant correlation matrix defined as:
\[
\mathbf{R}=\left[\begin{array}{cc}
1 & \rho\\
\rho & 1
\end{array}\right],
\]
hence we NO longer need the part regarding $\mathbb{\mathbf{Q}}_{t}$
in our estimation procedure. This assumption imply a factorization
of $\boldsymbol{\Sigma}_{t}$ according to $\mathbf{D}_{t}^{\frac{1}{2}}\mathbf{R}\mathbf{D}_{t}^{\frac{1}{2}}$,
hence our estimation procedure becomes marginally easier as we no
longer need to derive $\mathbf{R}_{t}$ for each time period. Instead
of filtering out the correlation matrix across time - as it is now
constant - one now only needs to filter out the covariances across
time, which may do by recalling the formula for correlation:
\[
\rho_{i;j}=\frac{\sigma_{i;j,t}}{\sigma_{i,t}\sigma_{j_{t}}}\Rightarrow\sigma_{i;j,t}=\rho_{i;j}\sigma_{i,t}\sigma_{j_{t}}.
\]

Note, this is obviously for the simplified case, in practice, we consider
the factorization and obtain
\[
\boldsymbol{\Sigma}_{t}=\mathbf{D}_{t}^{\frac{1}{2}}\mathbf{R}\mathbf{D}_{t}^{\frac{1}{2}}
\]
One may therefore realise that modelling the conditional variances
will be sufficient as they explain all variation in $\boldsymbol{\Sigma}_{t}$.\\

Considering the CCC model, it has been widely used in the litterature
due to its simplicity, but in general, it may face difficulties when
considering financial returns as the assumption of constant correlation
seldom is supported by data - one should note that due to the quite
high and quite constant (except for a few spikes) correlation among
the indexes we consider, GSPC and DJI, the CCC model will still perform
quite well.\\

Considering the constraints of the model, as it is also based on the
GARCH(1,1) marginal processes, we must require that (as we do with
the DCC model)
\begin{align*}
0\leq & \alpha,\beta<1, & \alpha+\beta & <1,
\end{align*}
in order to have weak stationarity. Furthermore, we have that if the
conditional variances, $\mathbf{D}_{t}$, is all positive, $\mathbf{R}$
is positive definite, then $\boldsymbol{\Sigma}_{t}$ will be guarenteed
to be positive definite.

\subsection*{DCC vs CCC}

In order to visualize the difference between the two models, one may
consider the following graph, which indeed describe the main difference
between the models, i.e. the correlation matrix for the two models,
i.e. $\mathbf{R}_{t}$ for DCC and $\mathbf{R}$ for CCC.

\begin{figure}[H]
\begin{centering}
\includegraphics[scale=0.4,bb = 0 0 200 100, draft, type=eps]{2019/DCC vs CCC.jpeg}
\par\end{centering}
\caption{DCC vs CCC model - Correlation}
\end{figure}

Hence as described in words; the DCC model will allow for the correlation
to vary across time, whereas the CCC model will keep correlation constant.

\section*{Empirical Analysis}

\subsection*{Data Series}

Note, as stated initially, for question 2, I have made use of the
updated dataset, i.e. data\_new.csv, which features the returns for
the indexes GSPC and DJI.\\

As an initial step, I have started by plotting the returns against
time and against each other.

\begin{figure}[H]
\begin{centering}
\subfloat[Returns against time]{\begin{centering}
\includegraphics[scale=0.4,bb = 0 0 200 100, draft, type=eps]{2019/Returns of the series.jpeg}
\par\end{centering}
} \subfloat[Returns against returns]{\begin{centering}
\includegraphics[scale=0.4,bb = 0 0 200 100, draft, type=eps]{2019/Returns plotted against each other.jpeg}
\par\end{centering}
}
\par\end{centering}
\caption{Index returns}
\end{figure}

As one instantly notices, the returns of the two indexes follow very
much the same pattern and considering returns against returns, one
clearly find a strong positive correlation between the indexes. This
seems quite intuitive - the indexes features a variety of companies
and if the indexes contain similar companies, one would have indexes
that would be positively correlated. This is indeed the case and considering
the correlation between the indexes over the entire time interval,
it is close to 1.\\

The strong positive correlation may help us in the following, as it
will allow us to form expectations about the results we obtain, including:
\begin{enumerate}
\item One would expect that the portfolio weights in the minimum variance
portfolio would vary quite a lot over time due to the postive correlation.
\item When computing the Conditional Value at Risk, CoVaR, it will (most
likely) be negative.
\end{enumerate}

\subsection*{Minimum Variance Portfolio (MVP)}

Considering the minimum variance portfolio, we seek to compute the
weight, $\omega_{t}$ associated to the MVP constructed using GSPC
and DJI at each point in time, i.e.
\[
y_{t}=\omega_{t}y_{t}^{GSPC}+\left(1-\omega_{t}\right)y_{t}^{DJI}.
\]

In the following, we have illustrated the weights in both indexes
for both models. Note that short selling as well as fractional holdings
of either index is allowed in our setting.

\begin{figure}[H]
\begin{centering}
\subfloat[Weight on GSPC, $\omega_{t}$]{\begin{centering}
\includegraphics[scale=0.4,bb = 0 0 200 100, draft, type=eps]{2019/MVP - Weight on GSPC.jpeg}
\par\end{centering}
}\subfloat[Weight on DJI, $1-\omega_{t}$]{\begin{centering}
\includegraphics[scale=0.4,bb = 0 0 200 100, draft, type=eps]{2019/MVP - Weight on DJI.jpeg}
\par\end{centering}
}
\par\end{centering}
\caption{Minimum Variance Portfolio}
\end{figure}

One notices instantly that short selling is used to a large extend
and that the portfolio weights vary a great deal over the time horizon.
It it is a bit difficult to state a clear pattern, but in the first
couple of years, our DCC model tends to short the GSPC index more
than the CCC model does - it tends to favour larger positions in DJI
than the CCC model. The pattern gets less clear as time goes by and
the two models look more similar in terms of weights from 2014 and
forward.\\

In general, one may conclude that the DCC model is extreme in terms
of the position in the indexes, which must be caused by the time-varying
correlation, i.e. the model is less restricted, hence become more
'extreme' in terms of portfolio weights.

\subsection*{CoVaR of GSPC given DJI}

As a last step, we woud like to obtain the Conditional Value at Risk
- CoVaR for our two models; DCC and CCC. The Conditional Value at
Risk is defined in the following way as in L14S4:
\[
P\left(\left.Y_{1,t}\leq CoVaR_{t}\left(\alpha\right)\right|Y_{2,t}\leq VaR_{t}\left(\alpha\right),\mathcal{F}_{t-1}\right)=\alpha,
\]
hence the size we compute is essentially the $\alpha\%$ chance of
returns below $VaR_{GSPC,t}\left(\alpha\right)$ conditioning on the
$\alpha\%$ chance of returns below $VaR_{DJI,t}\left(\alpha\right)$.
To clarify with an example; if the $CoVaR_{GSPC,t}=VaR_{DJI,t}$ is
a $2\%$ return, then $\alpha=0.01$ imply that we have $1\%$ chance
of realising returns smaller than or equal to 2\% on our GSPC index,
conditioning on the fact that we have a $1\%$ chance of realising
returns smaller than or equal to $2\%$ for the DJI index.\\

As we have already considered the data series and noted that the indexes
are very positive correlated, one would therefore naturally assume
that the $CoVaR$ is negative. Note, if the two indexes where uncorrelated,
$CoVaR_{GSPC,t}=VaR_{GSPC,t}$.\\

To take a step deeper into the theory, I have decided to follow the
approach by Giulio Girardi and A. Tolga Erg�n (Journal of Banking
\& Finance, Volume 37, Issue 8, August 2013, Pages 3169-3180).

Using their definition and reframing it according to our problem,
the CoVaR is defined as the VaR for index $i$, GSPC, conditional
on index $j$, DJI, being in 'financial distress', which will allow
to show some of the risk spillovers that may exists between the indexes.\\

In general we consider two versions of the $CoVaR$;
\begin{enumerate}
\item $CoVaR=$ which imply that the event we are conditioning on is \emph{exactly}
at its $VaR$, i.e. 
\[
P\left(\left.Y_{1,t}\leq CoVaR_{t}\left(\alpha\right)\right|Y_{2,t}=VaR_{t}\left(\alpha\right),\mathcal{F}_{t-1}\right)=\alpha,
\]

\begin{enumerate}
\item This version has been modelled at the end of the code (but is NOT
graphed below, if one would like to compare.
\end{enumerate}
\item $CoVaR=$ which imply that the event we are conditioning on is \emph{at
most} at its $VaR$, i.e. 
\[
P\left(\left.Y_{1,t}\leq CoVaR_{t}\left(\alpha\right)\right|Y_{2,t}\leq VaR_{t}\left(\alpha\right),\mathcal{F}_{t-1}\right)=\alpha.
\]

\begin{enumerate}
\item This version is graphed below.\\
\end{enumerate}
\end{enumerate}
In the following we will consider the second version and in order
to estimate the size, we follow the three-step procedure, which reads:
\begin{enumerate}
\item Derive $P\left(Y_{2,t}\leq VaR_{t}\left(\alpha\right)\right)=\alpha$
as it will constitute a upper bound in the integral at point 3.
\item Estimation of the bivariate model one consider; i.e. note it is the
bivariate gaussian DCC (and CCC) model.
\item Computation of $P\left(\left.Y_{1,t}\leq CoVaR_{t}\left(\alpha\right)\right|Y_{2,t}\leq VaR_{t}\left(\alpha\right),\mathcal{F}_{t-1}\right)=\alpha$
by the integral:
\[
\int_{-\infty}^{\text{CoVaR}_{t}\left(\alpha\right)}\int_{-\infty}^{\text{VaR}_{t}\left(\alpha\right)}pdf_{t}\left(x,y\right)dydx=\alpha^{2},
\]
which one derives by the uniroot procedure, where we solve for the
unknown $\text{CoVaR}_{t}\left(\alpha\right)$.
\end{enumerate}
The formal steps to get to the integral is from the fact that we have:
\begin{align*}
P\left(\left.Y_{1,t}\leq CoVaR_{t}\left(\alpha\right)\right|Y_{2,t}\leq VaR_{t}\left(\alpha\right),\mathcal{F}_{t-1}\right) & =\alpha\\
\frac{P\left(\left.Y_{1,t}\leq CoVaR_{t}\left(\alpha\right),Y_{2,t}\leq VaR_{t}\left(\alpha\right)\right|\mathcal{F}_{t-1}\right)}{\underset{=\alpha}{\underbrace{P\left(\left.Y_{2,t}\leq VaR_{t}\left(\alpha\right)\right|\mathcal{F}_{t-1}\right)}}} & =\alpha\\
P\left(\left.Y_{1,t}\leq CoVaR_{t}\left(\alpha\right),Y_{2,t}\leq VaR_{t}\left(\alpha\right)\right|\mathcal{F}_{t-1}\right) & =\alpha\underset{=\alpha}{\underbrace{P\left(\left.Y_{2,t}\leq VaR_{t}\left(\alpha\right)\right|\mathcal{F}_{t-1}\right)}}\\
\int_{-\infty}^{\text{CoVaR}_{t}\left(\alpha\right)}\int_{-\infty}^{\text{VaR}_{t}\left(\alpha\right)}pdf_{t}\left(x,y\right)dydx & =\alpha^{2}.
\end{align*}

We obtain the $CoVaR_{GSPC,t}\left(\alpha\right)$ for $\alpha\in[0.01,0.05]$
as:

\begin{figure}[H]
\begin{centering}
\subfloat[DCC model]{\begin{centering}
\includegraphics[scale=0.4,bb = 0 0 200 100, draft, type=eps]{2019/CoVaR - DCC.jpeg}
\par\end{centering}
}\subfloat[CCC model]{\begin{centering}
\includegraphics[scale=0.4,bb = 0 0 200 100, draft, type=eps]{2019/CoVaR - CCC.jpeg}
\par\end{centering}
}
\par\end{centering}
\caption{Conditional Value at Risk - GSPC Given DJI}
\end{figure}

As expected, the $CoVaR_{GSPC}$ is indeed negative for both models.
As one notices, the smaller the $\alpha$, i.e. percentage chance
of extreme event, the effect of that event becomes larger, which intuitive
makes great sense; the more extreme event one consider, the smaller
the probability for that event to occur. Hence moving from $\alpha=0.05$
to $\alpha=0.01$ simply shifts the curve downward.
\end{document}
